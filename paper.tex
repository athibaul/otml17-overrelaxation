\documentclass{article} % For LaTeX2e
\usepackage{nips14submit_e,times}
\usepackage{hyperref}
\usepackage{url}
\usepackage{amsmath}
\usepackage{amssymb}
\usepackage{amsthm}
\usepackage{mathtools}

%\documentstyle[nips14submit_09,times,art10]{article} % For LaTeX 2.09

\usepackage{cite}

% Math operators
\newcommand{\scal}[2]{\left\langle #1 , #2 \right\rangle}
\DeclareMathOperator{\IR}{\mathbb{R}}
\DeclareMathOperator*{\argmin}{argmin}
\DeclareMathOperator{\One}{\mathbbm{1}}
\DeclareMathOperator{\Ccal}{\mathcal{C}}
\DeclareMathOperator{\logsumexp}{logsumexp}
\DeclareMathOperator{\diag}{diag}
\newcommand{\norm}[1]{\left\lVert #1 \right\rVert}
\renewcommand{\epsilon}{\varepsilon}

\theoremstyle{plain}
\newtheorem{theorem}{Theorem}
\newtheorem{proposition}{Proposition}
\newtheorem{lemma}{Lemma}
\newtheorem{corollary}{Corollary}

\theoremstyle{definition}
\newtheorem{definition}{Definition}

\theoremstyle{remark}
\newtheorem{remark}{Remark}
\newtheorem{example}{Example}


\title{Over-relaxed Sinkhorn's Algorithm for Regularized Optimal Transport}


\author{
Alexis THIBAULT\\
\'Ecole Normale Sup\'erieure\\
Paris, France\\
\texttt{alexis.thibault@ens.fr}
}

% The \author macro works with any number of authors. There are two commands
% used to separate the names and addresses of multiple authors: \And and \AND.
%
% Using \And between authors leaves it to \LaTeX{} to determine where to break
% the lines. Using \AND forces a linebreak at that point. So, if \LaTeX{}
% puts 3 of 4 authors names on the first line, and the last on the second
% line, try using \AND instead of \And before the third author name.

\newcommand{\fix}{\marginpar{FIX}}
\newcommand{\new}{\marginpar{NEW}}

\nipsfinalcopy % Uncomment for camera-ready version

\begin{document}


\maketitle

\begin{abstract}
This article describes a method for quickly calculating the solution to the regularized optimal transport problem. It generalizes and improves upon the widely-used iterative Bregman projections algorithm (or Sinkhorn's algorithm). The idea is to over-relax the Bregman projection operators, allowing for faster convergence. In practice this corresponds to elevating the diagonal scaling factors to a given power, at each step of the algorithm.
\end{abstract}

\section{Iterative Bregman projections}

\subsection{Regularized optimal transport}

\[
K^\epsilon(\gamma) = \scal{c}{\gamma}
+ \epsilon \sum_{i,j} \gamma_{i,j} (\log(\gamma_{i,j}) - 1)
\]
\begin{equation}\label{eq:reg_ot_pb}
\gamma^* = \argmin_{\Ccal_1 \cap \Ccal_2} K^\epsilon(\gamma^*)
\end{equation}
Constraint sets:
\begin{align*}
A_1 &: \begin{cases}
\IR^{mn} \rightarrow \IR^m \\
(A_1 x)_i = \sum_j x_{i,j}
\end{cases} &
A_2 &: \begin{cases}
\IR^{mn} \rightarrow \IR^n\\
(A_2 x)_j = \sum_i x_{i,j}
\end{cases}
\end{align*}
\begin{align*}
\Ccal_1 &= \left\{ \gamma\in\IR^{mn} \mid A_1 \gamma = \mu \right\} &
\Ccal_2 &= \left\{ \gamma\in\IR^{mn} \mid A_2 \gamma = \nu \right\}
\end{align*}
Kullback-Leibler divergence:
\[ 
KL(\gamma,\xi) = \sum_{i,j} \gamma_{i,j} \left( \log \left( \frac{\gamma_{i,j}}{\xi_{i,j}} \right) -1  \right) + \sum_{i,j} \xi_{i,j}
\]
Bregman projection onto $\Ccal$:
\[
P_{\Ccal}(\xi) := \argmin_{\gamma \in \Ccal} KL(\gamma,\xi)
\]
\[\argmin_{\gamma \in \Ccal_1 \cap \Ccal_2} K^\epsilon(\gamma) = P_{\Ccal_1 \cap \Ccal_2} (e^{-c/\epsilon})\]
Optimal transport plan: Bregman projection of $\gamma^0 = e^{-c/\epsilon}$ onto $\Ccal_1 \cap \Ccal_2$.

\subsection{Sinkhorn's algorithm}

\begin{align*}
\gamma^0 &= e^{-c/\epsilon} &
\gamma^{k+1} = P_{\Ccal_2}(P_{\Ccal_1}(\gamma^k))
\end{align*}
Bregman's result: iterative Bregman projections onto $\Ccal_1$ and $\Ccal_2$ converge to a point in the intersection $\Ccal_1 \cap \Ccal_2$ \cite{bregman67}.

Hence this algorithm is correct:
\[\lim_{k\rightarrow +\infty} \gamma^k = P_{\Ccal_1 \cap \Ccal_2}(\gamma^0) = \gamma^*\]


\subsection{Numerical version}

Projections actually correspond to diagonal scalings.
\begin{align*}
P_{\Ccal_1}(\gamma) &= \diag(a) \gamma &
a &=  \frac{\mu}{A_1 \gamma} \\
P_{\Ccal_2}(\gamma) &= \gamma \diag(b) &
b &= \frac{\nu}{A_2 \gamma}
\end{align*}
(pointwise division).

In practice, simply store $(a^k, b^k)$:
\[
\gamma^k = \diag(a^k) \gamma^0 \diag{b^k}
\]
\begin{align*}
a^{k+1} &= \frac{\mu}{\gamma^0 b^k} &
b^{k+1} &= \frac{\nu}{^t \gamma^0 a^{k+1}} 
\end{align*}

Efficient implementation: matrix operations on GPU.


\section{Over-relaxed Sinkhorn's algorithm}

\subsection{Over-relaxed projections}

Let us define the $\omega$-over-relaxed projection operator by:
\begin{equation}\label{eq:def_or_proj}
\log P^\omega_{\Ccal_k}(\gamma) = (1-\omega) \log \gamma + \omega \log P_{\Ccal_k}(\gamma),
\end{equation}
where the logarithm is taken coordinate-wise.
Note that $P^0 = Id$, $P^1 = P$, and $P^2$ is an involution (this is non-trivial, and uses the fact that $\Ccal_k$ is an affine subspace).

\subsection{Lyapunov function}
Let $\gamma^*$ denote the solution of the regularized OT problem.
The function $F$ is defined as:
\begin{equation}\label{eq:lyapunov_function}
F(\gamma) = KL(\gamma^*, \gamma)
\end{equation}
We shall use it as a Lyapunov function, so that $F(\gamma^{k+1}) < F(\gamma^k)$ as long as the process has not converged.

Note that it is not necessary to know $\gamma^*$ in order to calculate the difference $F(\gamma) - F(P^\omega_{\Ccal_k}(\gamma))$, as shown by the following lemma.
\begin{lemma}\label{lemma:lyapunov_decrease}
	Take $\gamma$ in $\IR^{mn}_{+*}$. The decrease in value of the Lyapunov function can be calculated with the following formula:
	\begin{equation} \label{eq:kl_diff_scal}
	F(\gamma) - F(P^\omega_{\Ccal_k}(\gamma)) = 
	\scal{\mu}{\varphi_\omega \left(\frac{A_1 \gamma}{\mu}\right)},
	\end{equation}
	where
	\begin{equation}
	\varphi_\omega(x) = x(1-x^{-\omega}) - \omega e^x
	\end{equation}
	is a real function, applied coordinate-wise.
\end{lemma}
Thus, calculating the decrease in value of the function $F$ is relatively cheap and straightforward.

\subsection{New algorithm}

\begin{theorem}
	Let $\theta$ and $\omega$ be such that $1\le \theta < \omega < 2$. Let $U$ be the open set defined by:
	\begin{equation}\label{eq:open_set_U}
	U = \left\{
	\gamma \in \IR_{+*}^{nm} \mid
	F(P^\omega_{\Ccal_1}(\gamma)) < F(\gamma)
	\text{ and }
	F(P^\omega_{\Ccal_2}(P^\theta_{\Ccal_1}(\gamma))) < F(P^\theta_{\Ccal_1}(\gamma))
	\right\}.
	\end{equation}
	Set the initial value $\gamma^0 = e^{-c/\epsilon}$, and iterate:
	\begin{align*}
	\gamma^{k+1} =
	\begin{cases}
	P^\theta_{\Ccal_2}(P^\theta_{\Ccal_1}(\gamma^k)) & \text{if } \gamma \in U \\
	P_{\Ccal_2}(P_{\Ccal_1})(\gamma^k) & \text{otherwise.}
	\end{cases}
	\end{align*}
	Then the iterates $(\gamma^k)$ converge to $\gamma^*$.
\end{theorem}
{\color{red} $\longrightarrow$ Has to calculate 3 projections sometimes. This result is weaker than the one from my report, but easier to prove. Maybe we could avoid having $U$, and rather perform the test independently on each variable ? (Or we could find a cheaper test).}
\begin{remark}
	Given lemma \ref{lemma:lyapunov_decrease}, we can see that $\gamma^* \in U$. Therefore, iterations eventually all use the over-relaxed operators.
\end{remark}

\begin{proof}
	(New proof, using the Lyapunov function. Much shorter and cleaner than the one from my report.)
\end{proof}


\section{Local rate of convergence ?}

\section{Experimental results}
{\color{red} TODO}


\section*{Acknowledgments}


%\subsubsection*{References}

\bibliographystyle{apalike}
\bibliography{references}

\end{document}
